\nonstopmode{}
\documentclass[a4paper]{book}
\usepackage[times,inconsolata,hyper]{Rd}
\usepackage{makeidx}
\usepackage[utf8,latin1]{inputenc}
% \usepackage{graphicx} % @USE GRAPHICX@
\makeindex{}
\begin{document}
\chapter*{}
\begin{center}
{\textbf{\huge Package `lab4bis'}}
\par\bigskip{\large \today}
\end{center}
\begin{description}
\raggedright{}
\item[Type]\AsIs{Package}
\item[Title]\AsIs{Linear regression}
\item[Version]\AsIs{1.0}
\item[Date]\AsIs{2015-09-15}
\item[Author]\AsIs{Niclas Lovsj<c3><b6>, Maxime Bonneau}
\item[Maintainer]\AsIs{}\email{niclas.lovsjo@me.com}\AsIs{}
\item[Description]\AsIs{Multiple linear regression. Mimics the lm-fun in R.}
\item[License]\AsIs{GPL-2}
\item[Suggests]\AsIs{testthat, knitr, rmarkdown}
\item[VignetteBuilder]\AsIs{knitr}
\item[NeedsCompilation]\AsIs{no}
\end{description}
\Rdcontents{\R{} topics documented:}
\inputencoding{utf8}
\HeaderA{lab4bis-package}{Linear regression}{lab4bis.Rdash.package}
\aliasA{lab4bis}{lab4bis-package}{lab4bis}
\keyword{package}{lab4bis-package}
%
\begin{Description}\relax
Multiple linear regression. Mimics the lm-fun in R.
\end{Description}
%
\begin{Details}\relax

The DESCRIPTION file:

\Tabular{ll}{
Package: & lab4bis\\{}
Type: & Package\\{}
Title: & Linear regression\\{}
Version: & 1.0\\{}
Date: & 2015-09-15\\{}
Author: & Niclas Lovsjö, Maxime Bonneau   \\{}
Maintainer: & <niclas.lovsjo@me.com>\\{}
Description: & Multiple linear regression. Mimics the lm-fun in R.\\{}
License: & GPL-2\\{}
Suggests: & testthat,
knitr,
rmarkdown\\{}
VignetteBuilder: & knitr\\{}
}

Index of help topics:
\begin{alltt}
coef.linreg             coef
coef.ridgereg           coef ridgereg
lab4bis-package         Linear regression
linreg                  linreg
plot                    plot
pred                    pred
predict                 predict ridge
print ridge             print ridge
print.linreg            print
q                       Functions.
resid                   resid
ridgereg                ridgereg
visualize_airport_delays
                        visualize_airport_delays
\end{alltt}

\textasciitilde{}\textasciitilde{} An overview of how to use the package, including the most important \textasciitilde{}\textasciitilde{}
\textasciitilde{}\textasciitilde{} functions \textasciitilde{}\textasciitilde{}
\end{Details}
%
\begin{Author}\relax
Niclas Lovsjö, Maxime Bonneau   

Maintainer: <niclas.lovsjo@me.com>
\end{Author}
%
\begin{References}\relax
\textasciitilde{}\textasciitilde{} Literature or other references for background information \textasciitilde{}\textasciitilde{}
\end{References}
%
\begin{SeeAlso}\relax
\textasciitilde{}\textasciitilde{} Optional links to other man pages, e.g. \textasciitilde{}\textasciitilde{}
\textasciitilde{}\textasciitilde{} \code{\LinkA{<pkg>}{<pkg>}} \textasciitilde{}\textasciitilde{}
\end{SeeAlso}
%
\begin{Examples}
\begin{ExampleCode}
~~ simple examples of the most important functions ~~
\end{ExampleCode}
\end{Examples}
\inputencoding{utf8}
\HeaderA{coef.linreg}{coef}{coef.linreg}
%
\begin{Description}\relax
this function only separates the estimate of the coefficients apart.
\end{Description}
%
\begin{Usage}
\begin{verbatim}
## S3 method for class 'linreg'
coef(result)
\end{verbatim}
\end{Usage}
%
\begin{Arguments}
\begin{ldescription}
\item[\code{what}] results of the use of linreg function
\end{ldescription}
\end{Arguments}
%
\begin{Value}
a named vector of the estimate of the coefficients
\end{Value}
\inputencoding{utf8}
\HeaderA{coef.ridgereg}{coef ridgereg}{coef.ridgereg}
%
\begin{Description}\relax
this function only separates the estimate of the coefficients apart.
\end{Description}
%
\begin{Usage}
\begin{verbatim}
## S3 method for class 'ridgereg'
coef(result)
\end{verbatim}
\end{Usage}
%
\begin{Arguments}
\begin{ldescription}
\item[\code{what}] results of the use of linreg function
\end{ldescription}
\end{Arguments}
%
\begin{Value}
a named vector of the estimate of the coefficients
\end{Value}
\inputencoding{utf8}
\HeaderA{linreg}{linreg}{linreg}
%
\begin{Description}\relax
mimics the lm-function in R, i.e. makes linear regression.
\end{Description}
%
\begin{Usage}
\begin{verbatim}
linreg(formula, data)
\end{verbatim}
\end{Usage}
%
\begin{Arguments}
\begin{ldescription}
\item[\code{a}] "formula" of the form y \textasciitilde{} x+... and a dataframe
\end{ldescription}
\end{Arguments}
%
\begin{Value}
an object with class "linreg", i.e. that has attributes according to linear regression.
\end{Value}
%
\begin{Author}\relax
Niclas Lovsjö \& Maxime Bonneau
\end{Author}
%
\begin{References}\relax
https://en.wikipedia.org/wiki/Linear\_regression
\end{References}
\inputencoding{utf8}
\HeaderA{plot}{plot}{plot}
%
\begin{Description}\relax
plots two graphs with the values calculated by linreg function :
- one plots residuals vs fitted values
- the other one plots the scale location, ie the square root of the
absolute value of the residuals
\end{Description}
%
\begin{Arguments}
\begin{ldescription}
\item[\code{what}] results of the use of linreg function
\end{ldescription}
\end{Arguments}
%
\begin{Value}
the two plots described over
\end{Value}
\inputencoding{utf8}
\HeaderA{pred}{pred}{pred}
%
\begin{Description}\relax
this function predicts the values of the parameter we want to explain,
ie it calculates y hat for the vector of values in parameter
\end{Description}
%
\begin{Usage}
\begin{verbatim}
pred(result, ...)
\end{verbatim}
\end{Usage}
%
\begin{Arguments}
\begin{ldescription}
\item[\code{what}] results of the use of linreg function, and a numeric vector, which length is the
same as the number of coefficients.
\end{ldescription}
\end{Arguments}
%
\begin{Value}
a vector with the predicted values
\end{Value}
\inputencoding{utf8}
\HeaderA{predict}{predict ridge}{predict}
%
\begin{Description}\relax
this function predicts the values of the parameter we want to explain,
ie it calculates y hat for the vector of values in parameter
\end{Description}
%
\begin{Usage}
\begin{verbatim}
predict(result, ...)
\end{verbatim}
\end{Usage}
%
\begin{Arguments}
\begin{ldescription}
\item[\code{what}] results of the use of ridgereg function, and a numeric vector, which length is the
same as the number of coefficients.
\end{ldescription}
\end{Arguments}
%
\begin{Value}
a vector with the predicted values
\end{Value}
\inputencoding{utf8}
\HeaderA{print ridge}{print ridge}{print ridge}
\aliasA{print.ridgereg}{print ridge}{print.ridgereg}
%
\begin{Description}\relax
this function prints important results calculated by ridgereg function
\end{Description}
%
\begin{Usage}
\begin{verbatim}
## S3 method for class 'ridgereg'
print(X)
\end{verbatim}
\end{Usage}
%
\begin{Arguments}
\begin{ldescription}
\item[\code{what}] results of the use of ridgereg function
\end{ldescription}
\end{Arguments}
%
\begin{Value}
the call, ie the formula of the regression, and the estimate of the coefficients,
with their names
\end{Value}
\inputencoding{utf8}
\HeaderA{print.linreg}{print}{print.linreg}
%
\begin{Description}\relax
this function prints important results calculated by linreg function
\end{Description}
%
\begin{Usage}
\begin{verbatim}
## S3 method for class 'linreg'
print(X)
\end{verbatim}
\end{Usage}
%
\begin{Arguments}
\begin{ldescription}
\item[\code{what}] results of the use of linreg function
\end{ldescription}
\end{Arguments}
%
\begin{Value}
the call, ie the formula of the regression, and the estimate of the coefficients,
with their names
\end{Value}
\inputencoding{utf8}
\HeaderA{q}{Functions.}{q}
\keyword{datasets}{q}
%
\begin{Description}\relax
Uses a GET-verb to connect to val.se and downloads a xls-file, which is
converted and read by a read.xlsx-command into the actual data-file we need. From this
file it extracts the things we need. The shiny app then reads this by calling the
package Lab5, which only consists of this data.
\end{Description}
%
\begin{Usage}
\begin{verbatim}
q
\end{verbatim}
\end{Usage}
%
\begin{Arguments}
\begin{ldescription}
\item[\code{No}] params
\end{ldescription}
\end{Arguments}
%
\begin{Format}
\begin{alltt}List of 10
 $ url        : chr "http://www.val.se/val/val2014/statistik/2014_riksdagsval_per_kommun.xls"
 $ status_code: int 200
 $ headers    :List of 9
  ..$ date          : chr "Tue, 20 Oct 2015 11:34:07 GMT"
  ..$ server        : chr "Apache"
  ..$ last-modified : chr "Fri, 03 Oct 2014 06:18:22 GMT"
  ..$ etag          : chr "\bsl{}"10c358-1d400-5047eb32cc780\bsl{}""
  ..$ accept-ranges : chr "bytes"
  ..$ content-length: chr "119808"
  ..$ cache-control : chr "max-age=3600"
  ..$ expires       : chr "Tue, 20 Oct 2015 12:34:07 GMT"
  ..$ content-type  : chr "application/vnd.ms-excel"
  ..- attr(*, "class")= chr [1:2] "insensitive" "list"
 $ all_headers:List of 1
  ..$ :List of 3
  .. ..$ status : int 200
  .. ..$ version: chr "HTTP/1.1"
  .. ..$ headers:List of 9
  .. .. ..$ date          : chr "Tue, 20 Oct 2015 11:34:07 GMT"
  .. .. ..$ server        : chr "Apache"
  .. .. ..$ last-modified : chr "Fri, 03 Oct 2014 06:18:22 GMT"
  .. .. ..$ etag          : chr "\bsl{}"10c358-1d400-5047eb32cc780\bsl{}""
  .. .. ..$ accept-ranges : chr "bytes"
  .. .. ..$ content-length: chr "119808"
  .. .. ..$ cache-control : chr "max-age=3600"
  .. .. ..$ expires       : chr "Tue, 20 Oct 2015 12:34:07 GMT"
  .. .. ..$ content-type  : chr "application/vnd.ms-excel"
  .. .. ..- attr(*, "class")= chr [1:2] "insensitive" "list"
 $ cookies    :'data.frame':	0 obs. of  7 variables:
  ..$ domain    : logi(0) 
  ..$ flag      : logi(0) 
  ..$ path      : logi(0) 
  ..$ secure    : logi(0) 
  ..$ expiration:Classes 'POSIXct', 'POSIXt'  num(0) 
  ..$ name      : logi(0) 
  ..$ value     : logi(0) 
 $ content    : raw [1:119808] d0 cf 11 e0 ...
 $ date       : POSIXct[1:1], format: "2015-10-20 11:34:07"
 $ times      : Named num [1:6] 0 0.00521 0.00542 0.00553 0.04721 ...
  ..- attr(*, "names")= chr [1:6] "redirect" "namelookup" "connect" "pretransfer" ...
 $ request    :List of 7
  ..$ method    : chr "GET"
  ..$ url       : chr "http://www.val.se/val/val2014/statistik/2014_riksdagsval_per_kommun.xls"
  ..$ headers   : Named chr "application/json, text/xml, application/xml, */*"
  .. ..- attr(*, "names")= chr "Accept"
  ..$ fields    : NULL
  ..$ options   :List of 2
  .. ..$ useragent    : chr "libcurl/7.43.0 r-curl/0.9.3 httr/1.0.0"
  .. ..$ customrequest: chr "GET"
  ..$ auth_token: NULL
  ..$ output    : list()
  .. ..- attr(*, "class")= chr [1:2] "write_memory" "write_function"
  ..- attr(*, "class")= chr "request"
 $ handle     :Class 'curl_handle' <externalptr> 
 - attr(*, "class")= chr "response"
\end{alltt}
\end{Format}
%
\begin{Author}\relax
Maxime Bonneau, Niclas Lovsjö
\end{Author}
\inputencoding{utf8}
\HeaderA{resid}{resid}{resid}
%
\begin{Description}\relax
this function takes apart the residuals of the linear regression
calculated by linreg function
\end{Description}
%
\begin{Usage}
\begin{verbatim}
resid(X)
\end{verbatim}
\end{Usage}
%
\begin{Arguments}
\begin{ldescription}
\item[\code{what}] results of the use of linreg function
\end{ldescription}
\end{Arguments}
%
\begin{Value}
a vector with the resiudals
\end{Value}
\inputencoding{utf8}
\HeaderA{ridgereg}{ridgereg}{ridgereg}
\aliasA{formula}{ridgereg}{formula}
\keyword{datasets}{ridgereg}
%
\begin{Description}\relax
Uses ridge regression to fit a model.
\end{Description}
%
\begin{Usage}
\begin{verbatim}
formula
\end{verbatim}
\end{Usage}
%
\begin{Arguments}
\begin{ldescription}
\item[\code{lambda}] - hyperparameter to be tuned. if not specified, we set it to 0.

\item[\code{a}] "formula" of the form y \textasciitilde{} x+... and a dataframe
\end{ldescription}
\end{Arguments}
%
\begin{Format}
\begin{alltt}Class 'formula' length 3 Sepal.Length ~ Sepal.Width + Petal.Width
  ..- attr(*, ".Environment")=<environment: namespace:lab4bis> 
\end{alltt}
\end{Format}
%
\begin{Author}\relax
Niclas Lovsjö \& Maxime Bonneau
\end{Author}
%
\begin{References}\relax
ESLII: http://web.stanford.edu/\textasciitilde{}hastie/local.ftp/Springer/OLD/ESLII\_print4.pdf
\end{References}
\inputencoding{utf8}
\HeaderA{visualize\_airport\_delays}{visualize\_airport\_delays}{visualize.Rul.airport.Rul.delays}
%
\begin{Description}\relax
Plots the mean delay of flights for different airports by longitude and latitude using ggplot2
\end{Description}
%
\begin{Arguments}
\begin{ldescription}
\item[\code{no}] parameters
\end{ldescription}
\end{Arguments}
%
\begin{Author}\relax
Niclas Lovsjö \& Maxime Bonneau
\end{Author}
%
\begin{References}\relax
https://www.rstudio.com/wp-content/uploads/2015/02/data-wrangling-cheatsheet.pdf
https://cran.r-project.org/web/packages/nycflights13/nycflights13.pdf
\end{References}
\printindex{}
\end{document}
